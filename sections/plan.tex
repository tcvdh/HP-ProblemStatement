\section{Plan}

For this project, i have planned about 40 weeks of work, completing by October 2026. The work is
divided into four phases, aligned with the research questions defined in the previous sections.

\subsection{Phase 1: RQ1 - Definition and modeling (weeks 1-8)}
This phase will result in a finalized mathematical workload specification and a completed
literature review chapter.
\begin{itemize}
	\item \textbf{Weeks 1-2 (Literature Review):} Set up the OpenDC codebase to get familiar with the
	      simulator and create a local experiment to understand how ti works. Conduct a literature
	      review on Scaling laws for LLMs, to gather the mathematical formulas needed for compute
	      requirements.
	\item \textbf{Weeks 3-5 (Literature Review):} Analyze the recent systems papers (GoCkpt, Byte Dance) to
	      extract formulas for I/O latency and model the "Software Stragglers" in the homogeneous
	      clusters.
	\item \textbf{Weeks 6-8(Model Formalization):} Synthesize findings into a draft "Parametric LLM Workload
	      Model" specification. Formalize the mathematical equations for "Gradient Synchronization"
	      and "Checkpointing Overhead".
\end{itemize}
\subsection{Phase 2: RQ2 - Design and Implementation (weeks 9-20)}
This phase will result in a functional OpenDC prototype implementation capable of simulating a full
training run.
\begin{itemize}
	\item \textbf{Weeks 9-10 (Architecture Design):} Analyze the OpenDC workload interface and design the
	      class hierarchy for SyntheticLLMWorkload.
	\item \textbf{Weeks 11-14 (Core Logic Implementation):} Implement the basic compute logic (duration
	      based on FLOPs) and 3D parallelism, which simulates the data/tensor /pipeline stages.
	\item \textbf{Weeks 15-18 (Overhead Logic Implementation):} Implement the Gradient Synchronization
	      barrier logic and the checkpointing logic (I/O stall events based on the waste formulas).
	      Run initial validation experiment with a small dummy cluster to verify it's stable.
	\item \textbf{Weeks 19-20 (Refinement):} Refine the code for performance and work on the documentation
	      for the API.
\end{itemize}
\subsection{Phase 3: RQ3 - Validation and Experimentation (Weeks 21-32)}
This phase will result in a complete dataset of experimental results and finalized charts.
\begin{itemize}
	\item \textbf{Weeks 21-23 (Validation Experiments):} Configure the simulator to match the hardware
	      specifications of the BLOOM training cluster. Run simulations to compare energy/duration
	      against the public logs and calibrate the model parameters accordingly.
	\item \textbf{Weeks 24-27 (Experimentation and Execution):} Design and execute the full factorial
	      experiment. Run batches for a baseline asynchronous checkpointing run and an optimized
	      Gradient-Assisted Checkpointing run, using clusters of variable sizes (128 to 16,384
	      GPUs).
	\item \textbf{Weeks 28-32 (Data Processing):} Process raw output logs into usable datasets. Generate
	      initial visualizations like charts of throughput vs cluster size and advanced plots for
	      cost vs energy.
\end{itemize}
\subsection{Phase 4: Analysis and Writing (Weeks 33-40)}
\begin{itemize}
	\item \textbf{Weeks 33-35 (Drafting Results):} Draft the Methodology, Implementation and Results
	      chapters. Get started on a Discussion and Conclusion, focussing on the societal impact of
	      energy efficiency.
	\item \textbf{Weeks 36-37 (Review and Revision):} Submit the first full draft to supervisors and
	      revise the document based on feedback.
	\item \textbf{Weeks 38-40 (Final polishing):} Finalize formatting and references. Prepare the final presentation slides.
	\item \textbf{Week 40 (Submission):} Submit the final thesis document.
\end{itemize}

