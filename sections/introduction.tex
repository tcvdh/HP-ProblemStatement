
\section{Introduction} \label{sec:introduction}

Explain the research project. Also include here the personal value you hope to derive from this project.

Explain at least:
\begin{enumerate}
    \item The context of this research project. How broad do you see the impact of a good result? (Will you change the world? The science of Europe? The industry of the Netherlands?)
    
    \item The key terms addressed in this research project. You will expand on this element in Section~\ref{sec:background}.
    
    \item The main problem addressed in this research project. You will expand on this element in Section~\ref{sec:problem}.
    
    \item The key prior work related to this research project. You will expand on this element in Section~\ref{sec:related}.
    
    \item The main research question, possibly paraphrased. You will expand on this element in Section~\ref{sec:researchq}. (If possible, also indicate the core of the approach, or an insight that can lead to it. You will expand on this element in Section~\ref{sec:approach}.)
    
    \item The expected contribution of this research, for the scientific community and/or for your employer. You will expand on this element in Sections~\ref{sec:researchq}, \ref{sec:approach}, and~\ref{sec:plan}.
    
    \item Expected contribution of this research, for yourself. How will this project develop you? How will it develop your career?
    
\end{enumerate}

For example, consider the project leading to publication~\cite{DBLP:conf/sc/AndreadisVMI18}:
\begin{enumerate}
    \item Context: datacenters, the backbone of cloud computing and our digital economy.
    \item Key terms: datacenters, scheduling, reference architecture.
    \item Problem: understanding and improving the process of scheduling in datacenters.
    \item Key prior work: research on scheduling in large-scale systems, scheduling practices in Big Tech companies (Google, Microsoft, Alibaba, etc.)
    \item Main research question: How to design a good abstraction for datacenter scheduling? Key insight: a unified reference architecture is a good  abstraction for the scheduling process.
    \item Expected contribution, community: a survey, a reference architecture, an analysis of existing systems as mapped to the new reference architecture, a simulator implementing the reference architecture as the scientific instrument, experiments in simulation, description of a process for others to use the reference architecture, analysis of threats to validity. 
    Plus: a technical report accompanying the publication\footnote{The technical report is published as open science: \url{https://arxiv.org/pdf/1808.04224.pdf}}, various public talks, etc. (The team also went for and obtained the ACM reproducibility badge, which among others requires publishing FOS software and FAIR data.)
    \item Expected contribution, personal: development into an independent researcher.
\end{enumerate}
